% Options for packages loaded elsewhere
% Options for packages loaded elsewhere
\PassOptionsToPackage{unicode}{hyperref}
\PassOptionsToPackage{hyphens}{url}
\PassOptionsToPackage{dvipsnames,svgnames,x11names}{xcolor}
%
\documentclass[
  authoryear,
  review,
  3p]{elsarticle}
\usepackage{xcolor}
\usepackage{amsmath,amssymb}
\setcounter{secnumdepth}{5}
\usepackage{iftex}
\ifPDFTeX
  \usepackage[T1]{fontenc}
  \usepackage[utf8]{inputenc}
  \usepackage{textcomp} % provide euro and other symbols
\else % if luatex or xetex
  \usepackage{unicode-math} % this also loads fontspec
  \defaultfontfeatures{Scale=MatchLowercase}
  \defaultfontfeatures[\rmfamily]{Ligatures=TeX,Scale=1}
\fi
\usepackage{lmodern}
\ifPDFTeX\else
  % xetex/luatex font selection
\fi
% Use upquote if available, for straight quotes in verbatim environments
\IfFileExists{upquote.sty}{\usepackage{upquote}}{}
\IfFileExists{microtype.sty}{% use microtype if available
  \usepackage[]{microtype}
  \UseMicrotypeSet[protrusion]{basicmath} % disable protrusion for tt fonts
}{}
\makeatletter
\@ifundefined{KOMAClassName}{% if non-KOMA class
  \IfFileExists{parskip.sty}{%
    \usepackage{parskip}
  }{% else
    \setlength{\parindent}{0pt}
    \setlength{\parskip}{6pt plus 2pt minus 1pt}}
}{% if KOMA class
  \KOMAoptions{parskip=half}}
\makeatother
% Make \paragraph and \subparagraph free-standing
\makeatletter
\ifx\paragraph\undefined\else
  \let\oldparagraph\paragraph
  \renewcommand{\paragraph}{
    \@ifstar
      \xxxParagraphStar
      \xxxParagraphNoStar
  }
  \newcommand{\xxxParagraphStar}[1]{\oldparagraph*{#1}\mbox{}}
  \newcommand{\xxxParagraphNoStar}[1]{\oldparagraph{#1}\mbox{}}
\fi
\ifx\subparagraph\undefined\else
  \let\oldsubparagraph\subparagraph
  \renewcommand{\subparagraph}{
    \@ifstar
      \xxxSubParagraphStar
      \xxxSubParagraphNoStar
  }
  \newcommand{\xxxSubParagraphStar}[1]{\oldsubparagraph*{#1}\mbox{}}
  \newcommand{\xxxSubParagraphNoStar}[1]{\oldsubparagraph{#1}\mbox{}}
\fi
\makeatother


\usepackage{longtable,booktabs,array}
\usepackage{calc} % for calculating minipage widths
% Correct order of tables after \paragraph or \subparagraph
\usepackage{etoolbox}
\makeatletter
\patchcmd\longtable{\par}{\if@noskipsec\mbox{}\fi\par}{}{}
\makeatother
% Allow footnotes in longtable head/foot
\IfFileExists{footnotehyper.sty}{\usepackage{footnotehyper}}{\usepackage{footnote}}
\makesavenoteenv{longtable}
\usepackage{graphicx}
\makeatletter
\newsavebox\pandoc@box
\newcommand*\pandocbounded[1]{% scales image to fit in text height/width
  \sbox\pandoc@box{#1}%
  \Gscale@div\@tempa{\textheight}{\dimexpr\ht\pandoc@box+\dp\pandoc@box\relax}%
  \Gscale@div\@tempb{\linewidth}{\wd\pandoc@box}%
  \ifdim\@tempb\p@<\@tempa\p@\let\@tempa\@tempb\fi% select the smaller of both
  \ifdim\@tempa\p@<\p@\scalebox{\@tempa}{\usebox\pandoc@box}%
  \else\usebox{\pandoc@box}%
  \fi%
}
% Set default figure placement to htbp
\def\fps@figure{htbp}
\makeatother





\setlength{\emergencystretch}{3em} % prevent overfull lines

\providecommand{\tightlist}{%
  \setlength{\itemsep}{0pt}\setlength{\parskip}{0pt}}



 
\usepackage[]{natbib}
\bibliographystyle{elsarticle-harv}


\usepackage{booktabs}
\usepackage{longtable}
\usepackage{array}
\usepackage{multirow}
\usepackage{wrapfig}
\usepackage{float}
\usepackage{colortbl}
\usepackage{pdflscape}
\usepackage{tabu}
\usepackage{threeparttable}
\usepackage{threeparttablex}
\usepackage[normalem]{ulem}
\usepackage{makecell}
\usepackage{xcolor}
\makeatletter
\@ifpackageloaded{caption}{}{\usepackage{caption}}
\AtBeginDocument{%
\ifdefined\contentsname
  \renewcommand*\contentsname{Table of contents}
\else
  \newcommand\contentsname{Table of contents}
\fi
\ifdefined\listfigurename
  \renewcommand*\listfigurename{List of Figures}
\else
  \newcommand\listfigurename{List of Figures}
\fi
\ifdefined\listtablename
  \renewcommand*\listtablename{List of Tables}
\else
  \newcommand\listtablename{List of Tables}
\fi
\ifdefined\figurename
  \renewcommand*\figurename{Figure}
\else
  \newcommand\figurename{Figure}
\fi
\ifdefined\tablename
  \renewcommand*\tablename{Table}
\else
  \newcommand\tablename{Table}
\fi
}
\@ifpackageloaded{float}{}{\usepackage{float}}
\floatstyle{ruled}
\@ifundefined{c@chapter}{\newfloat{codelisting}{h}{lop}}{\newfloat{codelisting}{h}{lop}[chapter]}
\floatname{codelisting}{Listing}
\newcommand*\listoflistings{\listof{codelisting}{List of Listings}}
\makeatother
\makeatletter
\makeatother
\makeatletter
\@ifpackageloaded{caption}{}{\usepackage{caption}}
\@ifpackageloaded{subcaption}{}{\usepackage{subcaption}}
\makeatother
\journal{Journal of Cleaner Production}
\usepackage{bookmark}
\IfFileExists{xurl.sty}{\usepackage{xurl}}{} % add URL line breaks if available
\urlstyle{same}
\hypersetup{
  pdftitle={Resilient structure and increasing concentration of the roundwood trade: a network analysis perspective},
  pdfauthor={Valentin Mathieu; David Shanafelt},
  pdfkeywords={Globalization, Trade, Wood products, Network
theory, Forest policy},
  colorlinks=true,
  linkcolor={blue},
  filecolor={Maroon},
  citecolor={Blue},
  urlcolor={Blue},
  pdfcreator={LaTeX via pandoc}}


\setlength{\parindent}{6pt}
\begin{document}

\begin{frontmatter}
\title{Resilient structure and increasing concentration of the roundwood
trade: a network analysis perspective}
\author[1,2]{Valentin Mathieu%
\corref{cor1}%
}
 \ead{valentin.mathieu@agroparistech.fr} 
\author[2]{David Shanafelt%
%
}
 \ead{david.shanafelt@inrae.fr} 

\affiliation[1]{organization={Université de Lorraine, AgroParisTech,
INRAE,
Silva},addressline={54000},city={Nancy},country={France},countrysep={,},postcodesep={}}
\affiliation[2]{organization={Université de Lorraine, Université de
Strasbourg, AgroParisTech, CNRS, INRAE,
BETA},addressline={54000},city={Nancy},country={France},countrysep={,},postcodesep={}}

\cortext[cor1]{Corresponding author}


        
\begin{abstract}
{[}TO EDIT{]} While the use of wood products is increasingly linked to
sustainability and climate change issues, forest resources are unevenly
distributed among countries and have different regional forest dynamics.
Redistributing surpluses resulting from overproduction of wood products
to countries with demand deficits to address sustainability issues can
be achieved through trade. The literature offers a comprehensive study
of the international timber trade, but few studies to date consider the
physical structure of the trade network in their analyses. We conduct a
diagnosis of the international roundwood trade from 1997 to 2017 using a
network-theoretic approach, which provides a deeper understanding of the
inherent structure of the trade of wood products and complements the
existing literature. We expect the global roundwood trade network to be
concentrated and to follow global trends in economic growth, but also to
be sensitive to major economic and political events and natural
disasters. Our results show that the roundwood trade network has
increased in value, while varying slightly in size, and has become more
interconnected over the study period. We also observed changes in
network structure over time that can be explained by three broad
categories of events: economic disruptions, political events, and
natural disasters. Most importantly, our results show the growing market
power of China: in 2017, more than 50\% of the network's total trade
value was due to Chinese imports, China was among the most connected
countries, and countries preferentially traded with China. Lastly, our
results have implications for forest policy, as the structure of the
network may dictate potential cascades of local or regional shocks to
the global trade market, and as the growing polarization around China
exacerbates trade tensions in roundwood.
\end{abstract}





\begin{keyword}
    Globalization \sep Trade \sep Wood products \sep Network
theory \sep 
    Forest policy
\end{keyword}
\end{frontmatter}
    

\section{Introduction}\label{introduction}

International wood product trade stands as an increasingly complex and
interdependent component of the global economy, intricately tied to
environmental concerns, geopolitical shifts, evolving competition
schemes, and macroeconomic trends. The uneven global distribution of
forest resources, concentrated in a few countries (\emph{e.g.}, Russia,
Brazil, Canada, the United States, and China) \citep{fao2024state},
necessitates international trade to balance supply and demand to satisfy
domestic consumption \citep{long_exploring_2019, huang_static_2024}.

The 21st century has witnessed significant acceleration in international
wood product trade, driven by convergent economic and technological
determinants. Primary drivers include sustained global economic growth,
escalating demand for construction materials fueled by urbanization,
trade liberalization, and technological innovations enhancing forestry
value chain operational efficiency\footnote{These innovations encompass
  advancements in plantation management systems, forest harvesting
  operations, and manufacturing processes.}
\citep{Prestemon2003, 2021unecefao}. Furthermore, global value chains
(GVC) proliferation has restructured production paradigms through the
spatial fragmentation of manufacturing processes, leading to new market
dynamics and more complex supply chain configurations
\citep{amador2017networks}.

This trade intensification has precipitated systematic displacement of
deforestation pressures, characterized by distinct geographic
redistribution patterns. Resource accumulation occurs in nations with
predominantly non-indigenous, plantation-based forest systems ---
typically developed economies with established silviculture
infrastructure --- while simultaneously exerting extractive pressure on
regions where indigenous forest ecosystems face depletion, predominantly
within developing nations
\citep{Prestemon2003, leblois2017has, pendrill2019deforestation, abman2020does}.
This spatial divergence highlights how trade-mediated demand drives
forest degradation. In particular, illegal logging and associated trade
represent significant drivers of deforestation and forest degradation
globally, exacerbating climate change
\citep{lawson2010illegal}.\footnote{Illegal logging and its associated
  trade also extensively contribute to wider environmental damage, such
  as global anthropogenic carbon dioxide emissions, loss of
  biodiversity, deprive governments of billions of dollars in vital
  revenues, foster corruption, undermine the rule of law, and can fuel
  conflict \citep{lawson2010illegal}.}

This acceleration of the international wood trade has also coincided
with growing environmental consciousness and a rising preference for
``green'' materials with carbon storage capacity, driven by climate
change mitigation goals, leading to various environmental regulations
and sustainability initiatives \citep{Prestemon2003}. The emerging
forest-based bioeconomy, exemplified by strategies like the European
Bioeconomy Strategy, aims to replace non-renewable resources with
high-value-added products from woody biomass
\citep{wolfslehner2016forest, winkel2017towards}. This creates a
paradox: while increased wood demand can support decarbonization and
sustainable development, unsustainable and illegally sourcing risks
exacerbating deforestation and forest degradation. Consequently,
market-based tools like forest certification schemes\footnote{Such as
  the Forest Stewardship Council (FSC) and the Program for the
  Endorsement of Forest Certification (PEFC).} promote sustainable
practices, sometimes restricting wood products trade
\citep{guan2019restricting, chen2020effect, boubacar2025sustainable}.
Furthermore, timber trade regulations, primarily enforced by developed
countries, aim to exclude illegal timber from domestic markets,
hindering trade volumes
\citep{moral2020transparency, rougieux2021impacts, apeti2023impact, kim2024analyzing}.\footnote{Examples
  are the European Union's Action Plan for Forest Law Enforcement,
  Governance and Trade (FLEGT), the European Union Timber Regulation
  (EUTR), the US Lacey Act, the Australia Illegal Logging Prohibition
  Act, or the Japan Clean Wood Act.}

While tariff barriers have declined due to international trade
agreements, these trade regulations demonstrate that non-tariff barriers
(NTBs) increasingly influence wood product trade \citep{2021unecefao}.
NTBs include quotas, embargoes, economic sanctions, export bans,
stringent environmental policies, and phytosanitary controls
\citep{li2007potential, sun2010impacts, buongiorno2018potential, 2022unecefao, 2024fao}.
Notable trade disputes over wood products, such as those between China
and the U.S. \citep{muhammad2021end, pan2021impacts}, Russian roundwood
export restrictions
\citep{turner2008implications, solberg2010forest, lin2017incidence, guan2024impact},
and the Canada-U.S. softwood lumber dispute
\citep{van2014global, johnston2017impact}, underscore a tension between
free trade principles and national policy objectives. Such measures,
while often aiming for sustainability or industry protection, can
disrupt trade, raise prices, and shift production and consumption
patterns.

In recent years, significant economic and geopolitical disruptions have
affected the global supply chain for wood products, leading to
volatility in timber prices and shipping costs
\citep{2024fao, 2022unecefao}. The COVID-19 pandemic initially
introduced considerable uncertainty into forest product markets but was
swiftly followed by a robust economic rebound in 2021, indicating strong
underlying demand and inherent resilience \citep{2022unecefao}. However,
the war in Ukraine from mid-2022 exacerbated supply chain disruptions,
intensified inflation, and eroded consumer confidence, leading to a
demand decline \citep{2022unecefao} Climate change impact on forests
ecosystems worsens such instability, increasing the frequency and
severity of natural disturbances like storms, wildfires, and pest
outbreaks
\citep{seidl2011unraveling, seidl2017climate, curtis2018classifying, tyukavina2022global, patacca2023significant},
directly disrupting timber supply chains and market stability and
deteriorating wood quality \citep{garcia2025forest}. This apparent
vulnerability to external shocks suggests that trade resilience is not
absolute but contingent on global economic, environmental, and political
stability \citep{garcia2025forest, ma2025modelling}.

Understanding the intricate web of these relationships and trade flows
is paramount for policy formulation and strategic planning. The
literature extensively studies the international timber market and
trade, spanning from global to subnational scales and various wood
products \citep[\emph{e.g.},][
\citet{shen_structural_2022}]{muller2004long, raunikar2010global, caurla2013stimulating, van2014global, buongiorno2015global, rougieux2017modelling, rougieux2021impacts}.
These studies employ three primary perspectives: trade mechanism
analysis, forecasting, and policy analysis
\citep{buongiorno2016gravity, riviere2020representations, mathieu2023meta}.

Overall, the literature on international wood trade draws on models.
Traditional approaches, primarily forest sector models and econometric
models (including gravity models) rooted in economic theory
\citep{buongiorno1996forest, kallio2004global, latta2013review, northway2013forest, buongiorno2015global},
rely on nationally reported data and treat countries as the main unit of
analysis, focusing on discrete bilateral trade relationships
\citep{amador2017networks, shen_structural_2022}.

However, these approaches face significant limitations in accounting for
multilateral network effects, indirect trade linkages, and cascading
interdependencies inherent in modern international trade and
contemporary global value chain structures
\citep{de2011world, amador2017networks, liu2025spatial}. Consequently,
traditional approaches offer limited structural insights and may
systematically underestimate critical systemic vulnerabilities such as
supply chain fragility, contagion effects, and network resilience, while
simultaneously overlooking strategic opportunities revealed by
comprehensive network topology analysis
\citep{fevre2006animal, huang_static_2024}. Furthermore, the existing
literature exhibits considerable heterogeneity in analytical frameworks,
temporal specifications, and spatial delimitations, hindering systematic
meta-analysis and comprehensive understanding of global trade network
dynamics.

We address this gap by adopting a network theoretic approach that
considers the trade flows between countries as the unit of analysis
\citep{de2011world, amador2017networks}. This perspective offers a
deeper, holistic view of wood product trade inherent structure,
providing insights into its underlying organization and evolution
\citep{long_exploring_2019, shen_structural_2022}. Indeed, network
analysis has been a powerful tool for studying complex global
interactions.\footnote{See, for example, \citep{fevre2006animal}, who
  construct networks of the live animal trade to explain the spread of
  disease.}

Network analysis has significantly advanced the understanding of global
wood forest product trade dynamics, moving beyond traditional economic
models to reveal structural characteristics, intricate relationships,
competitive dynamics, evolutionary trends, network resilience to
external shocks, and vulnerabilities
\citep{lovric_social_2018, long_exploring_2019, gao_trade_2024, huang_static_2024}.
Previous research indicates these trade networks have significantly
evolved and expanded in the 21st century, showing increased complexity
and interdependence, broader participation, and enhanced overall trade
efficiency and resilience
\citep{wang_exploratory_2020, gao_trade_2024, huang_static_2024, liu_analysis_2024}.
While network density remains stable, average trade value has risen
\citep{zhou_spatial_2021}.

Network analysis has also identified key players and their evolving
roles. Historically, North American and European countries were central,
but emerging economies like China have become dominant importers and
increasingly central players across the supply chain, with their
influence growing annually
\citep{zhou_spatial_2021, gao_trade_2024, huang_static_2024, liu_analysis_2024}.
Russia and Canada remain major timber exporters, and New Zealand is a
significant raw material exporter \citep{zhou_spatial_2021}. Despite
dynamic changes, these networks maintain a relatively stable
core-periphery structure, with China and India now prominent in the core
\citep{wang_exploratory_2020}. While core countries boost network
resilience, their dominance introduces vulnerabilities
\citep{huang_static_2024}.

Beyond structural descriptions, the literature has also investigated
factors influencing network evolution, including economic scale,
geographical distance, cultural differences, and forest resource
endowment \citep{gao_trade_2024}, internal trade effects (reciprocity
and transitivity) \citep{shen_structural_2022}, and policy changes
\citep{huang_static_2024}.

Current research on forest product trade networks, while foundational,
needs deeper methodological and empirical exploration. Most studies take
a broad, macroscopic view, often combining different wood product types,
potentially distorting network understanding and leading to less
effective trade policies \citep{gao_trade_2024, liu_analysis_2024}. A
significant gap exists in comprehensively characterizing the structure
and temporal evolution of international roundwood trade networks. This
gap is critical because roundwood --- unprocessed timber directly
harvested from forests --- represents the primary commodity in global
forest product supply chains, directly linked to sustainable forest
management and deforestation issues. As the second most traded wood
product by volume over the last decade,\footnote{According to FAOSTAT
  data} roundwood forms the foundational stratum of international forest
product trade, linking forest resource extraction with processing
industries through complex international trade relationships. The lack
of detailed analysis on roundwood trade patterns, both structurally and
over time, is a major gap in forest economics research, limiting policy
insights for sustainable forest governance.

The present work addresses this gap through a comprehensive network
analysis of global roundwood trade dynamics spanning the period
1996--2022. First, we conduct a multidimensional network topology
analysis using complementary quantitative metrics to systematically
characterize roundwood trade network structural properties and their
temporal evolution. This employs graph theory-based methodologies
alongside traditional trade analysis metrics for comprehensive
structural characterization. Second, we use information of past market
behaviour to explain changes in network structure. We hypothesise that
global roundwood trade networks demonstrate systematic concentration in
exports (due to concentrated forest endowments) while exhibiting
structural sensitivity to exogenous disruption events, including
economic crises, geopolitical instabilities, and natural disasters. This
hypothesis framework incorporates three propositions: (1) network
concentration in exports remains moderately high over the study period;
(2) trade network demonstrates systematic vulnerability to major
disruption events; and (3) trade network demonstrates short-term
structural recovery patterns following disruption events.

\section{Material and methods}\label{material-and-methods}

\subsection{Bilateral trade data
collection}\label{bilateral-trade-data-collection}

We extracted trade data from the UN Comtrade database in Python 3.12.2
using the Python package ``comtradeapicall'' 1.2.1
\citep{2024comtradeapicall}.\footnote{The Python package
  ``comtradeapicall'' extracts and downloads UN Comtrade data through
  API calls.} The UN Comtrade data provides detailed information on
bilateral trade flows by recording the trade reports from almost 200
countries. National statistical reports of trade may be produced on an
annual or monthly basis. They cover a wide range of traded commodities,
including wood products. Traded commodities are commonly classified
under the Harmonized System (HS) nomenclature \citep{1983hs}. The HS
nomenclature encodes each commodity under a 6-digit code, classifying
them into chapters (2 digits), headings (4 digits), and subheadings (6
digits). Each trade flow is described through a set of information,
including:

\begin{itemize}
\tightlist
\item
  the reporter, \emph{i.e.}, the country that reports the trade flow;
\item
  the partner, \emph{i.e.}, the country with which the reporter trades;
\item
  the period of reporting, \emph{i.e.}, year of trade and, if reported
  monthly, month of trade;
\item
  the type of trade flow, \emph{i.e.}, import, export, re-import,
  re-export
\item
  the product HS code and description
\item
  the trade flow's net weight and traded value (in current US dollars)
\end{itemize}

Since its creation, the HS nomenclature has been updated approximately
every five years. It currently comprises six editions (1996, 2002, 2007,
2012, 2017 and 2022). Each update to the HS nomenclature may result in
changes to commodity codes. For instance, new products may be added,
some products may be removed due to, for example, low trade volume, or a
product may be specified more precisely and divided into several
products.\footnote{For instance, between the 2007 and the 2012 HS
  nomenclature editions, the product ``Sawdust and wood waste and scrap,
  whether or not agglomerated in logs, briquettes, pellets or similar
  forms'' (code 440130) was subdivided into ``Wood pellets'' (code
  440131) and ``Other sawdust and wood waste and scrap, whether or not
  agglomerated in logs, briquettes, pellets or similar forms'' (code
  440139).} Although there are corresponding tables between different
editions of the HS nomenclature, tracking its evolution is difficult,
and changes in product nomenclature can result in gaps in time series.

Furthermore, although the HS nomenclature provides a common language for
trade in all commodities between more than 200 countries, it can lack
relevance to industry sectors. Wood products, for example, may be
grouped together with bamboo materials or plastic furniture in a single
chapter. To ensure consistency, we base our extraction on the
sector-specific 2022 FAO Classification of Forest Products
\citep{2022FAOclassif}, which provides corresponding tables between the
FAO classification and the 2017 and 2022 HS nomenclatures. We extract
the corresponding HS product codes for FAO code 012, ``Wood in the rough
other than wood fuel'' (hereafter referred to as ``roundwood''), and we
ensure consistency between different HS codes across the successive
editions of the HS nomenclature.\footnote{Data were extracted on June
  2025.} The correspondence between FAO product code 012 and HS product
codes across successive HS nomenclature editions is provided in
supplementary material X. All trade flows are then aggregated by HS
product codes, enabling the analysis to focus on roundwood trade flows
as defined by the FAO's Classification of Forest Products. This approach
ensures the sectorial relevance and continuity of the time series for
the analysis.

As country reports in the UN Comtrade database can be updated or
corrected within two years, we collect trade data from 1996 to 2022 to
base our analysis on consolidated data. We only extract data on imports
and exports. We also remove poorly specified trade flows, \emph{e.g.},
those with a ``non-elsewhere specified'' (NES) partners. Finally, to
reduce the noise in the data and remove unrealistic trade
flows,\footnote{For example, unrealistic trade flows typically
  correspond to trade flows with a net weight of a few kilograms.} we
remove the 5\% of trade flows with the lowest net weight or value from
the dataset.

\subsection{Trade network building from bilateral trade
data}\label{trade-network-building-from-bilateral-trade-data}

Based on this data, we define the roundwood trade network for each year
by computing edge lists. In network theory, an edge list is a
mathematical representation of a network, which is made up of nodes
(\emph{e.g.}, countries, individuals, or entities such as banks or
websites) and edges or vertices (\emph{e.g.}, the movement of goods or
people between countries, physical interactions between people, or the
exchange of information or money)
\citep{albert2002statistical, newman2003structure}. An example of an
edge list for a simple network is provided in
Figure~\ref{fig-edge-list-unweighted}. In our case, the edge lists are
\(N \times 2\) matrices that correspond to a roundwood trade network
with \(N\) edges in year \(t\). Each edge represents a trade flow
(either export or import) between nodes (countries), from the country of
origin (\emph{i.e.}, the exporter, indexed \(o\)), to the country of
destination, (\emph{i.e.}, the importer, indexed \(d\)), in year
\(t \in {1996,1998,…,2022}\). The elements of the edge list are nodes
that represent countries that engaged in either the export or import of
roundwood, in a given year \(t\).

\begin{figure}[t]

\centering{

\includegraphics[width=1\linewidth,height=\textheight,keepaspectratio]{figures/fig-edge-list-unweighted.png}

}

\caption{\label{fig-edge-list-unweighted}Illustration of an edge list,
\(E^U\), and its corresponding network graph for an \textit{unweighted}
and \textit{directed} network of 5 nodes. In the edge list, each row
corresponds to an edge from an origin (first column) to a destination
(second column). In the network graph, circles indicate nodes and arrows
the edges between them.}

\end{figure}%

We take the roundwood trade network for each year of the data as a
weighted and directed network, \emph{i.e.}, we consider whether or not
two countries are connected by trade, the direction of the trade flows,
and the ``weight'' (or quantity/value) of the trade flows. In our case,
we consider the traded value to be the ``weight'' of the trade flow.
This is in contrast to using the net weight in metric tons, which is not
an appropriate unit of volume for roundwood as the mass of the roundwood
depends on its moisture content.\footnote{UN Comtrade data provides an
  alternative ``quantity'' variable (encoded \emph{qty})to the net
  weight of the trade flows. However, this variable is poorly reported:
  (i) units may differ depending on the reporter and range from cubic
  metres to metres to kilograms, and so on; (ii) the ``quantity''
  variable is not recorded in 7.8\% of the dataset.} Due to data
discrepancies, the reported trade value from the exporter and importer
of the same trade flow may differ for at least two reasons: (1) import
reports are expressed as ``cost, insurance and freight'' (CIF), while
export reports are expressed as ``free on board'' (FOB), which excludes
CIF costs from trade value, leading to bilateral asymmetries; (2) one of
the trading partners may not report the trade
\citep{CEPII:2010-23, rougieux2017forest, kallio2018reliability, chen2022advancing, mitikj2024bridging}.
To minimise the bias that such inconsistencies may cause, we consider
both the exporter's and importer's reports of trade: \(a_{o,d}\) and
\(a_{d,o}\), respectively. The edge list corresponding to the weighted,
directed trade network is now a matrix of size
\(N \times 4\).\footnote{In fact, edge lists are close to mirror flows
  that can be derived from bilateral trade data, such as that provided
  by the UN Comtrade database.} Each row represents a trade flow for a
given year \(t\), including the country of origin, the country of
destination, and the trade values reported by the exporter and importer,
as shown in Figure~\ref{fig-edge-list-weighted}.\footnote{See
  \citet{rayfield2011connectivity} and \citet{thompson2017loss} for
  examples of this approach in ecological networks.} We therefore define
28 edge lists to describe the roundwood trade network for each year of
trade.

\begin{figure}[t]

\centering{

\includegraphics[width=1\linewidth,height=\textheight,keepaspectratio]{figures/fig-edge-list-weighted.png}

}

\caption{\label{fig-edge-list-weighted}Illustration of an edge list,
\(E^W\), and its corresponding network graph for a \textit{weighted} and
\textit{directed} trade network of 5 nodes. In the edge list, each row
corresponds to an edge from an origin (first column) to a destination
(second column) and include trade values reported by the country of
origin (third column, red) and the country of destination (fourth
column, blue). In the network graph, circles indicate nodes and arrows
the edges between them.}

\end{figure}%

\subsection{Trade network properties
assessment}\label{trade-network-properties-assessment}

Edge lists and any other network objects or raw visualisations are often
difficult to read and interpret. Alone, they do not enable a thorough
analysis of the structure of a trade network. In order to assess the
structural characteristics of trade networks, we computed a series of
network metrics for each year, differentiating between exports and
imports.\footnote{Considering directed networks enables analysis from
  either the origin (exporter) or destination (importer) country
  perspective, allowing exporter or importer trade behaviour to be
  inferred.} The Python package ``networkx'' 3.4.2
\citep{SciPyProceedings_11} is used to facilitate the measurement
network metrics.

First, we assess the composition of the trade network. For each year, we
compute the total number of nodes in the network, \emph{i.e.}, the total
number of trading countries. These countries are then divided into three
more detailed groups: pure exporters (countries that only export
roundwood), pure importers (countries that only import roundwood) and
mixed countries (countries that both export and import roundwood). To
obtain more detailed information on the composition of the network, we
then cross-reference the export and import values of each trading
country accounting for at least 1\% of to the global export or import
value with the number of its export and import trading partners. This
provides country trade profiles for each year, allowing trade behaviours
and groups to be inferred and their evolution observed over time. This
set of metrics provides a general overview of the roundwood trade
network.

Secondly, to understand the finer details of the network structure, we
then computed a set of metrics that assess the connectivity of the trade
network, \emph{i.e.}, the mean, variance, skewness, and kurtosis of the
number of connections per node
\citep{albert2002statistical, newman2003structure}. Again, we assess
network connectivity differentiating exports and imports. The mean
number of connections per node, or node degree, is an average indicator
of the degree of connectedness of the network. A high mean number of
connections per node means that each country in the network has, on
average, a high number of trading partners. The variance of the number
of connections per node provides additional information about the
dispersion of the number of connections. If the variance is close to
zero, then the nodes in the network have more or less the same degree. A
high variance indicates that the degree of connectivity of countries
largely deviates from the mean. Skewness provides information about the
proportion of low to highly connected nodes in the network. If the
skewness is negative, the network consists of a higher proportion of
highly connected nodes. Conversely, if the skewness is positive, then
the network has a higher proportion of lowly connected nodes. Zero
skewness implies an equal ratio of low to high connected nodes. Kurtosis
provides information on how the degree distribution compares with a
normal distribution in terms of its ``tailedness'' or ``flatness''. A
positive kurtosis\footnote{We consider Fisher's definition of kurtosis.
  A Pearson's definition would have necessitated the comparison of
  kurtosis values with 3 instead of 0.} indicates that tailed outliers
are more prevalent than in a normal distribution, meaning countries that
diverge from the mean are rare. Conversely, negative kurtosis
corresponds to a situation in which outliers are common, \emph{i.e.},
where most countries diverge from the mean.

Lastly, we assess market concentration by combining our network approach
with a traditional Herfindahl-Hirschman market concentration index
(HHI). The HHI is defined as follows:

\begin{equation}\phantomsection\label{eq-HHI}{
HHI = \sum^{N}_{i = 1} (MS_{i})^{2}
}\end{equation}

where \(N\) is the number of countries involved in trade and \(MS_{i}\)
the market share of the country \(i\), that is, the ratio of the value
traded by country \(i\) to the global value traded. We computed two
\(HHI\), one for exports and one for imports. The \(HHI\) values can
range from \(1/N\) to 1. According to the \(HHI\) value, the US Federal
Trade Commission differentiates three types of markets
\citep{doj2010horizontal}:

\begin{itemize}
\tightlist
\item
  An \(HHI\) between 0.01 and 0.15 corresponds to an unconcentrated
  market, \emph{i.e.}, a market with no anti-competitive effects
  presumed;
\item
  An \(HHI\) between 0.15 and 0.25 indicates a moderately concentrated
  market, \emph{i.e.}, a market with potential or significant
  anti-competitive effects;
\item
  An \(HHI\) greater than 0.25 indicates a highly concentrated market,
  \emph{i.e.}, a market with presumption of anti-competitive effects.
\end{itemize}

To obtain a more detailed understanding of market concentration, we
calculated each country's annual contribution to the global trade value,
\emph{i.e.}, the reduction in total trade value if a given country were
removed from the network. This is equivalent to estimating the share of
total trade value that ``flows'' through a given country in a particular
year. Consequently, the sum of all countries' contributions does not add
up to 100\%, due to the bilateral nature of trade flows. This approach
allows us to identify dominant countries in the roundwood trade. In
cases of high market concentration, the removal of a dominant country is
expected to result in a substantial loss in trade value.

Combining multiple complementary metrics offers a more comprehensive
understanding of the network's structure than any single metric alone
can provide \citep{shanafelt2017yourself, salau2022taking}. For example,
networks may have the same average number of connections but exhibit
vastly different variances, resulting in markedly different topologies.
Moreover, we consider our selected set of metrics sufficient for
analyzing trade network structure, as many network metrics tend to be
correlated \citep{baggio2011landscape}. The metrics used and their
interpretations are summarized in Table~\ref{tbl-network-metrics}.

\begin{table}

\caption{\label{tbl-network-metrics}List of the metrics used to describe
the trade network and their interpretation.}

\centering{

\begin{tabu} to \linewidth {>{\raggedright}X>{\raggedright}X}
\toprule
Network metrics & Interpretation\\
\midrule
\addlinespace[0.3em]
\multicolumn{2}{l}{\textbf{Trade network composition}}\\
\hspace{1em}Number of nodes & Number of countries involved in trade.\\
\hspace{1em}Number of pure exporter & Number of countries that only export roundwood.\\
\hspace{1em}Number of pure importer & Number of countries that only import roundwood.\\
\hspace{1em}Number of mixed countries & Number of countries that both export and import roundwood.\\
\hspace{1em}Country profiles & Trade behaviour of countries in relation to exports and imports.\\
\addlinespace[0.3em]
\multicolumn{2}{l}{\textbf{Trade network connectivity}}\\
\hspace{1em}Mean number of connections per node & Average number of trading partners per country; indicator of the overall connectedness of the network.\\
\hspace{1em}Variance in the number of connections per node & Variation in the number of trading partners relative to the mean.\\
\hspace{1em}Skewness in the number of connections per node & Proportion of low to highly connected countries in the network.\\
\hspace{1em}Kurtosis in the number of connections per node & Scarcity or abundance of countries that diverge from the mean connectivity.\\
\addlinespace[0.3em]
\multicolumn{2}{l}{\textbf{Market concentration}}\\
\hspace{1em}Herfindahl-Hirschman index & Market concentration in the trade of roundwood.\\
\hspace{1em}Country contributions to trade value & Share of total trade value that 'flows' through a given country.\\
\bottomrule
\end{tabu}

}

\end{table}%

\subsection{Computational workflow and
reproducibility}\label{computational-workflow-and-reproducibility}

A sustainable data analysis workflow was undertaken using Snakemake
9.6.0 \citep{molder2021sustainable}. Snakemake is a workflow management
system that ensures the reproducibility, adaptability, and transparency
of the data analysis. This workflow is available on a GitHub
repository,\footnote{GitHub repository address:
  \url{https://github.com/vlmathieu/trade_network_analysis}} which
provides access to the analysis and version control
\citep{braga2023not}.

\section{Results}\label{results}

\subsection{Composition of the trade network over
time}\label{composition-of-the-trade-network-over-time}

Over the study period, the number of countries involved in the roundwood
trade fluctuated slightly, ranging from 189 in 1997 to 214 in 2008 and
2015 (Figure~\ref{fig-network-composition}). On average, 206 countries
were involved in roundwood trade from 1996 to 2022. After a moderate
increase in the number of countries involved in trade between 1996 and
2008 (an increase of 11.46\%, from 192 to 214 countries), the number of
countries stabilised until 2015. From 2015 to 2022, the number of
trading countries decreased slightly from 214 to 198 (an 8.08\%
decrease). Aside from these trends, we found short-term slight decreases
in the number of trading countries after 2008, 2015, 2019, and 2021,
which, as we will discuss later, can likely be attributed to major
global events. In addition, we can observe several trends in the share
of pure exporters, pure importers, and countries that both export and
import roundwood. The number of pure exporters has decreased markedly
since 1996, falling from 17 to 7 (a decline of around 59\%). In 2022,
pure exporters represented only 3\% of trading countries compared to an
average of 6.2\% over the studied period. In contrast, the number of
pure importers decreased from 51 to 38 between 1996 and 2007 (a decline
of 25\%), before rising again to 60 between 2007 and 2015 (an increase
of 58\%). Since 2015, there has been a moderate decrease to 44 countries
(a fall of 27\%). In 2022, pure importers represented a larger share
(21\%) of trading countries than pure exporters (23.2\% on average over
the study period). Throughout the years, there were consistently more
pure importers than pure exporters engaged in the international trade of
roundwood, with an average difference of 35 countries. The number of
countries that both export and import roundwood (``mixed countries'')
constituted the vast majority of trading countries (70.6\% on average
over the studied period, 69\% in 2022). The number of mixed countries
has increased moderately since 1996, rising from 124 to 147 (an increase
of around 19\%). As for the total number of trading countries, we
noticed short terms drops in the number of pure exporters, pure
importers, and mixed countries over the studied period. Some drops are
common to pure importers and mixed countries (\emph{e.g.}, following
2008 and 2019), while others are group-specific (\emph{e.g.}, following
1997 for pure exporters and 2001 for pure importers).

\begin{figure}[t]

\centering{

\includegraphics[width=1\linewidth,height=\textheight,keepaspectratio]{figures/fig-network-composition.png}

}

\caption{\label{fig-network-composition}Number of trading countries in
total and per group: pure exporters (red), pure importers (blue), mixed
countries (grey); for each year of trade.}

\end{figure}%

While 200 countries were involved in the global roundwood trade in 2020,
only 31 countries (15.5\% of the total) contributed to at least 1\% of
the export or import value (Figure~\ref{fig-countries-profiles}). This
suggests a certain degree of market concentration in the trade of
roundwood. These 31 countries showed different behaviours, and we
identified roughly three groups. First, Asian countries, namely China,
India, Japan, the Republic of Korea and Vietnam, form a group of
importers. Importers have a higher trade value for imports than for
exports and have diversified their import connections more than their
export connections. Second, several countries belong to the group of
exporters. Their export value is higher than their import value, they
have a higher number of export partners compared to their import
partners, but their number of export trading partners varies. While
Papua New Guinea only export to a limited number of 13 countries, net
exporters such as Australia, New Zealand, Congo, Cameroon, Brazil and
the Russian Federation, have moderately diversified their number of
export trading partners (between 20 and 45). New Zealand,the USA, and
the Russian Federation stand out as the world's three largest suppliers
of roundwood. In particular, the USA shows a unique export pattern with
a high diversification of its export trading partners (95 export
connections). Finally, we can draw a third group of countries that tend
to export as much as they import in trade value, with a balanced number
of exporting and importing partners. This group is mainly made up of
European countries but also includes countries such as Canada and
Malaysia.

The trade situation has changed significantly between 2000 and 2020
(Figure~\ref{fig-countries-profiles}). In 20 years, China has overtaken
Japan as the main importer of roundwood. Japan's import value decreased
dramatically from more than \$2.33 billion in 2000 to about \$562
million in 2020, a decrease of 76\%. On the other hand, China's import
value increased considerably from more than \$1.654 billion in 2000 to
more than \$8.376 billion in 2020, an increase of about 406\%.
Surprisingly, the number of China's export trading partners increased
sharply by about 133\% between 2000 and 2020 (32 new export trading
partners), while the value of Chinese export trade decreased by more
than \$1.2 million on the same period (a decline of about 16\%), which
seems unlikely. In terms of trade value, New Zealand has emerged as the
world's top exporter of roundwood, while maintaining a relatively stable
number of trading partners. Although the USA and the Russian Federation
remained major exporters of roundwood, they appeared to be less
integrated into global roundwood trade. Between 2000 and 2020, the USA
has lost a large share of its trading partners to import (a fall of
42\%), while slightly reducing its export diversification (a fall of
only 8\%). Similarly, the Russian Federation lost 14 export trading
partners (25\%) and 6 import trading partners (46\%) in the same period.

\begin{figure}[p]

\centering{

\includegraphics[width=0.8981\linewidth,height=0.9\textheight]{figures/fig-countries-profiles.png}

}

\caption{\label{fig-countries-profiles}Bubble graph showing the profiles
of trading countries in 2020 (top) and in 2000 (bottom). Countries are
displayed according to the number of partners with whom the trade
exports and imports. The size of the bubbles increases with the value of
trade. Blue bubbles refer to imports and red bubbles refer to exports.
Only countries accounting for at least 1\% of the value of exports or
imports in 1996 or 2022 are shown.}

\end{figure}%

\subsection{Connectivity of the trade network over
time}\label{connectivity-of-the-trade-network-over-time}

\begin{figure}[t]

\centering{

\includegraphics[width=1\linewidth,height=\textheight,keepaspectratio]{figures/fig-network-connectivity.png}

}

\caption{\label{fig-network-connectivity}Mean number (top left),
variance (top right), skewness (bottom left), and kurtosis (bottom
right) of the connections per node. Metrics concerning exports are
plotted in red, metrics concerning imports are plotted in blue. Smooth
curve is based on a Loess function to highlight trends in the metrics.}

\end{figure}%

Results for mean, variance, skewness, and kurtosis of the connections
per country are presented in Figure~\ref{fig-network-connectivity}.
Three general observations can be made regarding the average number of
connections per country over time. Firstly, both exporters and importers
of roundwood observed an increase in their mean connectivity during the
first half of the study period (from 1996 to 2008). During this period,
mean connectivity of exporters increased by 25\% for exporters (from
12.4 to 15.4 connections per exporter), while that of importers
increased by 30\% (from 10 to 12.9 connections per importer). The
average number of connections per country then stabilised for exporters.
Conversely, mean connectivity of importers decreased by 9\% (from 12.9
to 11.7 connections per importer) from 2008 to 2016 and has since
stabilised. Secondly, exporters were, on average, 22\% more connected
than importers throughout the study period (14.3 connections per
exporter versus 11.7 connections per importer). The gap widened between
the first and second halves of the study period (17\% between 1996 and
2008, 27\% between 2008 and 2022). Thirdly, aside from general trends,
short-term increases and decreases in mean connectivity were observed
from year to year, with some shared by both by exporters and importers.
For example, mean connectivity of both exporters and importers dropped
in the years following 2008, 2016, and 2019.

The variance in the number of connections per node provides additional
information on how the connectivity of each country diverges from the
overall average. Overall, we observe higher variance in connectivity per
country for exporters and importers. Exporters and importers comprise
countries with a wide range of trading partners, some of which have a
significantly lower number of connections than average (poorly connected
countries), while others have a significantly higher number of
connections than average (highly connected countries). Variance in the
connectivity per country increased for importers from 1996 to 2003 and
for exporters until 2010, before decreasing until 2022. Between 1996 and
2005, variance in connectivity per country was, on average, higher for
importers than for exporters. This suggests that, during this time
period, importers showed greater spread in connectivity compared to the
mean. After 2005, exporters and importers displayed similar variances in
connectivity. Similar short-term increases and decreases to those
observed in mean connectivity per node were seen in the variance from
year to year.

Skewness provides information about the distribution of highly and
poorly connected nodes within a network. Skewness in connections per
country is always positive for both exporters and importers, indicating
a greater proportion of countries with low to high connectivity. It
remained stable on average for both exporters and importers between 1996
and 2005, followed by an increase until 2010 for exporters and until
2013 for importers. Until 2010, the degree of skewness in connectivity
was similar for exporters and importers. Since 2010, the skewness in
connectivity has become, on average, higher for importers than for
exporters, has slightly increased for importers and has decreased
moderately for exporters. That is, there was a greater proportion of
poorly connected countries among importers than among exporters, which
supports and complements our findings regarding the mean number of
connections.

Kurtosis indicates whether countries diverge from the mean connectivity
in a scarce or abundant manner. During the study period, kurtosis of
connections per country remained positive for both exporters and
importers. This suggests that countries whose connectivity diverges from
the mean connectivity are rarer than those whose connectivity is close
to the mean. From 1996 to 2014, skewness of connectivity was higher for
exporters than for importers, meaning that exporters diverging from the
mean connectivity were, on average, rarer than importers. The reverse
was true after 2014. Until 2022, kurtosis of connectivity decreased,
while remaining positive. This fall is steeper for exporters than for
importers. As with our other network metrics, we observed several
short-term fluctuations in skewness and kurtosis over time from year to
year, though these were less pronounced.

\subsection{Market concentration in roundwood trade over
time}\label{market-concentration-in-roundwood-trade-over-time}

\begin{figure}[t]

\centering{

\includegraphics[width=1\linewidth,height=\textheight,keepaspectratio]{figures/fig-market-concentration.png}

}

\caption{\label{fig-market-concentration}Herfindahl-Hirschman index
values over time. Index related to exports are displayed in red, those
related to imports in blue.}

\end{figure}%

Between 1996 and 1998, the Herfindahl-Hirschman Index (HHI) initially
fell for both exports (a decline of 38\%, from 0.18 to 0.11) and imports
(a decline of 52\%, from 0.21 to 0.1) indicating a rapid shift from
moderate to low market concentration in roundwood trade
(Figure~\ref{fig-market-concentration}). This decline was followed by an
increase in the HHI for exporters until 2006, reaching a HHI of 0.15,
before falling again until 2011 and reaching a HHI of 0.08. Market of
roundwood exports then remained stable and lowly concentrated until
2022, with an average HHI of 0.08 since 2011. In contrast, market
concentration of roundwood imports increased sharply, becoming
moderately concentrated after 2009 (HHI of 0.17) and highly concentrated
after 2013 (HHI of 0.25). After a steep decline from a HHI of 0.26 to
0.18 between 2015 and 2016, market concentration level recovered the
following year to reach a HHI of 0.35. From 2017 to 2021, imports
remained highly concentrated, with an average HHI of 0.35, before
falling to 0.26 in 2022. On average, over the last decade, imports have
remained highly concentrated, whereas exports have remained lowly
concentrated for 26 years.

\begin{figure}[p]

\centering{

\includegraphics[width=0.6\linewidth,height=0.9\textheight]{figures/fig-network-contribution.png}

}

\caption{\label{fig-network-contribution}Contribution of different
countries to the total traded value of the network. Only countries that
have contributed to more than 5\% of the total traded value at least
once during the study period are shown. The envelope corresponds to the
uncertainty in the reported values. Contributions are displayed by group
of countries: importers (top); mixed countries (middle); and exporters
(bottom).}

\end{figure}%

Contributions from individual countries to the total network trade value
provide a more detailed picture of market concentration
(Figure~\ref{fig-network-contribution}). These contributions correspond
to the proportion of the network's total trade value that would be lost
if a particular country is omitted from the network, \emph{i.e.}, the
proportion of total trade value flowing through that country. Over the
study period, only ten countries contributed at least 5\% to the
network's total trade value once. Referring to
Figure~\ref{fig-countries-profiles}, main contributors to the total
trade value can be divided into three groups: importers (China, India
and Japan); European countries (Germany, Czechia and Austria); and
traditional wood exporters (Canada, New Zealand, the Russian Federation,
and the USA).

Among importers, Japan's share of total trade value decreased
considerably from around 44\% in 1996 to less than 7\% in 2022. On the
other hand, India's contribution increased steadily from around 2\% in
1996 to around 16\% in 2012, after decreasing to 5\% in 2022. Most
notably, China's share of the total trade value increased sharply, from
around 4\% in 1996 to more than 56\% in 2021, before decreasing to 45\%
in 2022. Together with results from Figures \ref{fig-countries-profiles}
and \ref{fig-market-concentration}, this indicated that around 56\% of
the total trade value flowed through China in 2021 due to its imports,
which is consistent with the high level of market concentration in
roundwood imports in over the past decade.

Among European countries, on average, the contribution of Germany and
the Czech Republic increased steadily from around 6\% and 2\% in 1996 to
around 14\% and 8\% in 2022, respectively. In contrast, Austria's
contribution rose only slightly from around 4\% to 7\% over the same
period. These levels of contribution remain relatively low compared to
those of main importers such as China.

Among exporters, the contribution of the USA significantly decreased
over the studied period, falling from 33\% in 1996 to 13\% in 2007. It
then stabilized at an average of around 13.7\% over the 2007-2022
period. Canada's contribution to total trade value increased from 5\% in
1996 to 11\% in 2002, before decreasing to 6\% in 2007. It has remained
stable since then, averaging around 6.2\%. By contrast, New Zealand's
contribution initially slightly fell from 7\% in 1996 to 4\% in 1998,
then stabilised at an average of around 4.4\% between 1998 and 2008,
before rising significantly to 18\% in 2022. On the other hand, Russia
exhibits a unique pattern of varying contributions to the total trade
value throughout the studied period. Initially, Russian contributions
increased from 13\% in 1996 to 31\% in 2007, before dropping sharply to
12\% in 2012. This figure then stabilised at an average of around 11.4\%
between 2012 and 2018, before dropping again to below 1\% in 2022. This
correlates with findings presented in
Figure~\ref{fig-countries-profiles}, which suggests that Russian trade
integration has decreased over time. Except for Russia, which appears to
be influenced by short-term shocks, all contributors to the roundwood
trade exhibit long-term variations in their contribution.

\section{Discussion}\label{discussion}

Contrary to our expectations, the global roundwood trade network
demonstrated significant resilience to exogenous disruptions. Rather
than major short-term shifts, the network continued its long-term trend
toward a greater concentration and polarization around China, whose
market power has grown substantially in recent decades.

We found that while the number of countries participating in the
roundwood trade network remained relatively stable, its composition
changed significantly, with countries engaging more in both imports and
exports (Figures \ref{fig-network-composition},
\ref{fig-countries-profiles}). The network's connectivity tightened at
the turn of the 21st century (Figure~\ref{fig-network-connectivity}), a
trend consistent with broader globalization \citep{Prestemon2003}.

Our analysis of network connectivity further revealed a consistent
structure: a few highly-connected countries (the hubs) dominate trade, a
pattern that sharply diverges from the network's low overall mean
connectivity (Figure~\ref{fig-network-connectivity}). These hubs play a
critical role in the network's functionality \citep{huang_static_2024},
and potentially have huge market power. Many of the major hubs in the
international market, such as the USA, Canada, Western Europe,
Scandinavia, and Russia, maintained a high number of trading partners
throughout the study period.

However, our findings also highlight a significant shift in the trade
landscape over recent decades. We observed a decline among traditional
core importers and exporters, including Japan, Russia, and the USA,
concurrent with the rise of new key players like China and New Zealand
(Figures \ref{fig-countries-profiles}, \ref{fig-network-contribution}).
China, in particular, has become \emph{the} central player in the
roundwood trade, substantially increasing its number of import and
export partners. By 2021, China's imports accounted for 56\% of the
total roundwood trade value, leading to a highly concentrated and
polarized import market (Figures \ref{fig-market-concentration},
\ref{fig-network-contribution}). This confirms findings from other
studies
\citep{long_exploring_2019, zhou_spatial_2021, shen_structural_2022, huang_static_2024}.

This could appear as a paradox: despite forest resources being
geographically concentrated in a few countries, the roundwood export
market is not similarly concentrated. Instead, the import market is
highly concentrated, primarily due to demand from China, which
paradoxically also possesses the fifth-highest forest endowment and
presents the first-highest average annual net gain in forest area
\citep{fao2024state}. Indeed, due to Chinese government efforts in
spending, as well as China's market and policy reforms driven by forest
conservation demand, China has operated an drastic expansion of its
forestry and planted forest extensive margins over the last decades
\citep{demurger2009forest, zhang2019china, Zhao_2022}. However, despite
an increase in China's forest area and volume, low economic tree
planting efficiency and poor forest management have resulted in low
stand quality and growth rates
\citep{hoffmann2018adapting, hou2019intensifying, zhang2019china}. As a
result, domestic timber consumption --- primarily for industrial and
construction uses --- has outpaced domestic production, making China
highly dependent on imported wood products, which now account for
approximately half of its total timber supply
\citep{demurger2009forest, he2011projection, hoffmann2018adapting, hou2019intensifying}.

This overall suggests that roundwood trade flows are predominantly
demand-driven rather than supply-pushed. In the case of China, the
primary use of wood for industrial and construction uses
\citep{hou2019intensifying} echoes the role of population and
urbanization in shaping demand for materials, which has been identified
in the literature
\citep{2021unecefao, mathieu2023meta, villamor2024preparing}. Explain a
bit more, suggest that this will mechanically attract wood trade flows
to regions that are increasing in population and getting urbanised, such
as Africa and Asia. Explain that while these regions are (becoming)
centers of gravity for wood trade flows, the role of countries may
shift. In particular, China seems to initiate a decline in 2022
(Figure~\ref{fig-network-contribution}) which could announce the
beginning of new phase in the development of the international trade of
roundwood. Such decline may be relative to the building of more
comprehensive domestic supply chains or to structural mutation, such as
the ageing of the Chinese population and the industrial crisis.

Our findings reveal that the network's excessive reliance on China is a
key feature, as its demand drives trade flows and influences global
roundwood resource allocation \citep{huang_static_2024}. China's market
power is significant enough to exacerbate disruption risks for
less-connected countries, redirect trade flows in a competitive
environment, and foster trade co-dependence. In fact, China serves as
the main trading partner for most major exporters, making them highly
reliant on its demand. For example, in 2020, China imported 46\% of
roundwood exports from the USA, 67\% from Russia, 77\% from Congo, 78\%
from Papua New Guinea, and 86\% from New Zealand. The case of New
Zealand perfectly illustrates this co-dependence. Its rise as a major
exporter is closely tied to China's increasing demand. The share of New
Zealand's roundwood exports to China surged from just 1.2\% in 1996 to
89.3\% in 2022. This increasing co-dependence raises concerns about a
potential ``after-China'' era for these exporters
\citep{villamor2024preparing}: where would their roundwood go if not to
China? To mitigate risks from potential disruptions caused by hubs like
China or to limit co-dependence, countries can diversify their trade
relationships \citep{huang_static_2024}. The United States appears to
have pursued this strategy throughout the study period, as shown in our
results (Figure~\ref{fig-countries-profiles}).

While we expected the network to exhibit structural sensitivity to
exogenous disruption events, we found that the roundwood trade network
was rather resilient to such disruptions and has followed long-term
trends over the past decades. Certainly, the network structural metrics
showed low to moderate sensitivity to shocks and disruptions. The number
of pure exporters, pure importers, and mixed countries, showed slight
short-term drops over the studied period (Figure 3). Short-term
increases and decreases in connectivity metrics were also observed from
year to year. Yet, such shocks seems to only impact the network
structure on the short-term,

``From 2002 to 2021, China became a superpower in importing upstream
wood forest products, ranking first in import volume {[}\ldots{]} This
achievement indicates that China was highly dependent on wood imports
and had strong resistance and recovery capabilities to deal with
shocks.''

By considering multiple network metrics, we obtain a clearer picture of
the evolution of the network's structure than using a single metric
alone \citep{shanafelt2017yourself, salau2022taking}. The mean number of
connections gives us a global measure of network connectivity, but tells
nothing of the distribution of connections throughout the network. By
measuring variance, skewness, and kurtosis we observed changes in the
variation of connections per node, the proportion of poorly to
highly-connected nodes, and the scarcity or abundance of countries that
diverge from the mean connectivity over time. As a complement to
traditional market concentration indexes, measuring a country's
contribution to traded value provides a detailed overview of each
trader's relative market power. It is crucial to recognize that network
analysis should not be viewed as a replacement for traditional methods
but rather as a powerful complement. It offers a macro-level, systemic
perspective that can inform and enrich the micro-level insights derived
from econometric models or the aggregate flow analyses from gravity
models. Such multi-methodological approach would allow to leverage the
strengths of each technique, leading to a more holistic and robust
understanding of complex trade phenomena.

However, our results show that, despite the powerful explanatory
capacity of network analysis, the quality of the underlying trade data
is paramount. The validity and interpretability of the results are
directly influenced by the accuracy, completeness, and consistency of
the data, as pointed out in the literature
\citep{lovric_social_2018, wang_exploratory_2020, zhou_spatial_2021, huang_static_2024}.
While network analysis partly deals with data discrepancies by
simultaneously taking into account exports and imports, it reveals data
discrepancies such as inconsistent reporting, missing values, and
outliers \citep{kallio2018reliability, chen2022advancing}. Hence, our
results showed that China exported to 32 more countries in 2020 compared
to 2000, while simultaneously reporting a decrease of \$1.2 million in
export value on the same period (a decline of about 16\%), which seems
unlikely (Figure~\ref{fig-countries-profiles}). This further highlights
the caution that should be brought by modelers and data scientists when
analysing bilateral trade data, while emphasizes the need for methods to
process trade data and correct quality issues or for harmonized and
consistent data {[}\citet{CEPII:2010-23}; rougieux2017forest{]}.

Similarly, our results suggest that the country-level perspective is not
always the most relevant in trade analysis. In particular, results
highlight the singular trade balance of many European countries, which
export as much roundwood as they import. This would suggest to rather
consider the European Union as a whole, in order to get a clear vision
that is not blurred by individual European countries behaviours and that
will better reflect the trade weight of the European continent.

Network analysis capture complexity and describe its organisation but
further exploration of the determinants and dynamics are necessary to
understand what rule complexity organisation. ``To comprehensively
analyze the resilience of the global wood and forest products trade
network, future research is required to refine and improve the
construction of indicator systems.''

\section{Conclusion}\label{conclusion}

\textbf{Declaration of generative AI and AI-assisted technologies in the
writing process} During the preparation of this work the author(s) used
{[}NAME TOOL / SERVICE{]} in order to {[}REASON{]}. After using this
tool/service, the author(s) reviewed and edited the content as needed
and take(s) full responsibility for the content of the publication.

\section*{References}\label{references}
\addcontentsline{toc}{section}{References}

\renewcommand{\bibsection}{}
\bibliography{mybibfile.bib}





\end{document}
